\let\negmedspace\undefined
\let\negthickspace\undefined
\documentclass[journal]{IEEEtran}
\usepackage[a5paper, margin=10mm, onecolumn]{geometry}
\usepackage{lmodern} % Ensure lmodern is loaded for pdflatex
\usepackage{tfrupee} % Include tfrupee package

\setlength{\headheight}{1cm} % Set the height of the header box
\setlength{\headsep}{0mm}     % Set the distance between the header box and the top of the text
\usepackage{float}
\usepackage{gvv-book}
\usepackage{gvv}
\usepackage{cite}
\usepackage{amsmath,amssymb,amsfonts,amsthm}
\usepackage{algorithmic}
\usepackage{graphicx}
\usepackage{textcomp}
\usepackage{xcolor}
\usepackage{txfonts}
\usepackage{listings}
\usepackage{enumitem}
\usepackage{mathtools}
\usepackage{gensymb}
\usepackage{comment}
\usepackage[breaklinks=true]{hyperref}
\usepackage{tkz-euclide} 
\usepackage{listings}
\usepackage{gvv}                                        
\def\inputGnumericTable{}                                 
\usepackage[latin1]{inputenc}                                
\usepackage{color}                                            
\usepackage{array}                                            
\usepackage{longtable}                                       
\usepackage{calc}                                             
\usepackage{multirow}                                         
\usepackage{hhline}                                           
\usepackage{ifthen}                                           
\usepackage{lscape}
\begin{document}

\bibliographystyle{IEEEtran}
\vspace{3cm}
	
\title{2021 March 18 Shift 1}
\author{EE24BTECH11005 - Arjun Pavanje}
% \maketitle
% \newpage
% \bigskip
{\let\newpage\relax\maketitle}
\begin{enumerate}
	\setcounter{enumi}{15}
	\item If $lim_{x\to 0}\frac{\sbrak{\sin^{-1}x-\tan^{-1}x}}{3}$ is equal to $L$, then the value of $\brak{6L+1}$ is,
		\begin{enumerate}
				\begin{multicols}{2}
					\item $\frac{1}{2}$
				\columnbreak
					\item $2$
				\end{multicols}
				\begin{multicols}{2}
					\item $\frac{1}{6}$
				\columnbreak
					\item $6$
				\end{multicols}
		\end{enumerate}
	\item For all four circles $M,N,O,P$, the following four equations are given,
		\begin{align*}
			&\text{Circle M: }x^2+y^2=1\\
			&\text{Circle N: }x^2+y^2-2x=0\\
			&\text{Circle O: }x^2+y^2-2x-2y+1=0\\
			&\text{Circle P: }x^2+y^2-2y=0
		\end{align*}
		If the centre of circle $M$ is joined with the centre of circle $N$, furthur centre of circle $N$ is joined with centre of circle $O$, centre of circle $O$ is joined with centre of circle $P$ and lastly, the centre of circle $P$ is joined with the centre of circle $M$, then these lines form the sides of a,
		\begin{enumerate}
				\begin{multicols}{2}
					\item Rectangle
				\columnbreak
					\item Square
				\end{multicols}
				\begin{multicols}{2}
					\item Parallelogram
				\columnbreak
					\item Rhombus
				\end{multicols}
		\end{enumerate}
	\item Let $\brak{1+x+2x^2}^{20}=a_0+a_1x+a_2x^2+\dots+a_{40}x^{40}$. Then, $a_1+a_3+a_5+\dots+a_{37}$is equal to,
		\begin{enumerate}
				\begin{multicols}{2}
				\item $2^{20}\brak{2^{20}+21}$
				\columnbreak
			\item $2^{19}\brak{2^{20}+21}$
				\end{multicols}
				\begin{multicols}{2}
				\item $2^{20}\brak{2^{20}-21}$
				\columnbreak
			\item $2^{19}\brak{2^{20}-21}$
				\end{multicols}
		\end{enumerate}
	\item Let ,
		\begin{align*}
			&A+2B=\myvec{1&2&0\\6&-3&3\\-5&3&1}\\
			&2A-B=\myvec{2&-1&5\\2&-1&6\\0&1&2}
		\end{align*}
		If $tr\brak{A}$ denotes the sum of all diagonal entries of the matrix $A$, then $tr\brak{A}-tr\brak{B}$ is,
		\begin{enumerate}
				\begin{multicols}{2}
					\item $0$
				\columnbreak
					\item $1$
				\end{multicols}
				\begin{multicols}{2}
					\item $2$
				\columnbreak
					\item $3$
				\end{multicols}
		\end{enumerate}
	\item The equation of one of the straight lines which pass through the point$\myvec{1\\3}$ and make an angle $\tan^{-1}\sqrt{2}$ with the straight line $y+1=3\sqrt{2}x$ is,
		\begin{enumerate}
				\begin{multicols}{2}
				\item $5\sqrt{2}x+4y-15+4\sqrt{2}=0$
				\columnbreak
			\item $4\sqrt{2}x-5y-5+4\sqrt{2}=0$
				\end{multicols}
				\begin{multicols}{2}
				\item $4\sqrt{2}x+5y-4\sqrt{2}=0$
				\columnbreak
			\item $4\sqrt{2}x+5y-\brak{15+4\sqrt{2}}=0$
				\end{multicols}
			\end{enumerate}
	\item The number of times digit $3$ will be written when listing the integers from $1$ to $1000$ is \rule{2cm}{0.1pt}
	\item The equation of the planes parallel to the plane $x-2y+2z-3=0$ which are at unit distace from the point $\brak{1,2,3}$ is $ax+by+cz+d=0$. If $\brak{b-d}=k\brak{c-a}$, then the positive value of $k$ is\rule{2cm}{0.1pt}
	\item Let $f\brak{x},g\brak{x}$ be two functions satisfying $f\brak{x^2}+g\brak{4-x}=4x^3$ and $g\brak{4-x}+g\brak{x}=0$, then the value of $\int_{-4}^4 f\brak{x^2} dx$ is,\rule{2cm}{0.1pt}
	\item The mean age of $25$ teachers in a school is $40$ years. A teacher retires at the age of $60$ years and a new teacher is appointed in his place. If the mean age of the teachers in this school now is $39$ years, then the age of the newly appointed teacher is\rule{2cm}{0.1pt}
	\item A square $ABCD$ has all its vertices on the curve $x^2y^2 = 1$. The midpoints of its sides also lie on the same curve. Then, the square of the area of $ABCD$ is\rule{2cm}{0.1pt}
	\item The missing value in the following figure is,\rule{2cm}{0.1pt}
		\begin{figure}[H]
			\centering
			\includegraphics[scale=1]{figs26/fig26.png}
			\label{stemplot}
		\end{figure}
	\item The number of solutions of the equation $\abs{\cot x}=\cot x +\brak{\frac{1}{\sin x}}$ in the interval $\sbrak{0, 2\pi}$ is\rule{2cm}{0.1pt}
	\item Let $z_1,z_2$ be the roots of the equations $z_2+a_z+12=0$ and $z_1,z_2$ form an equilateral triangle with origin. Then, the value of $\abs{a}$ is \rule{2cm}{0.1pt}
	\item Let the plane $ax+by+cz+d=0$ bisect the line joining the points $\myvec{4\\-3\\1},\myvec{2\\3\\-5}$ at right angles. If $a,b,c,d$ are integers, then the minimum value of $\brak{a^2+b^2+c^2+d^2}$ is,\rule{2cm}{0.1pt}
	\item If $f\brak{x}=\int\frac{\sbrak{5x^8+x^6}}{\sbrak{x^2+1+2x^7}^2}dx, \brak{x\ge0}$, $f\brak{0}=0$ and $f\brak{1}=\frac{1}{k}$, then the value of $k$ is, \rule{2cm}{0.1pt}
\end{enumerate}
\end{document}
