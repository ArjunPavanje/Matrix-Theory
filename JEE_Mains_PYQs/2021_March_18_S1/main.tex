\let\negmedspace\undefined
\let\negthickspace\undefined
\documentclass[journal]{IEEEtran}
\usepackage[a5paper, margin=10mm, onecolumn]{geometry}
\usepackage{lmodern} % Ensure lmodern is loaded for pdflatex
\usepackage{tfrupee} % Include tfrupee package

\setlength{\headheight}{1cm} % Set the height of the header box
\setlength{\headsep}{0mm}     % Set the distance between the header box and the top of the text
\usepackage{float}
\usepackage{gvv-book}
\usepackage{gvv}
\usepackage{cite}
\usepackage{amsmath,amssymb,amsfonts,amsthm}
\usepackage{algorithmic}
\usepackage{graphicx}
\usepackage{textcomp}
\usepackage{xcolor}
\usepackage{txfonts}
\usepackage{listings}
\usepackage{enumitem}
\usepackage{mathtools}
\usepackage{gensymb}
\usepackage{comment}
\usepackage[breaklinks=true]{hyperref}
\usepackage{tkz-euclide} 
\usepackage{listings}
\usepackage{gvv}                                        
\def\inputGnumericTable{}                                 
\usepackage[latin1]{inputenc}                                
\usepackage{color}                                            
\usepackage{array}                                            
\usepackage{longtable}                                       
\usepackage{calc}                                             
\usepackage{multirow}                                         
\usepackage{hhline}                                           
\usepackage{ifthen}                                           
\usepackage{lscape}
\begin{document}

\bibliographystyle{IEEEtran}
\vspace{3cm}
	
\title{2021 March 18 Shift 1}
\author{EE24BTECH11005 - Arjun Pavanje}
% \maketitle
% \newpage
% \bigskip
{\let\newpage\relax\maketitle}
\begin{enumerate}
	\setcounter{enumi}{15}
\item If the functions are defined as $f\brak{x}=\sqrt{x}$ and $g\brak{x}=\sqrt{1-x}$, then what is the common domain of the following functions:\\
	$f+g,f-g,\frac{f}{g},\frac{g}{f},g-f$ where $\brak{f\pm g} \brak{x}=f\brak{x} \pm g\brak{x}, \brak{\frac{f}{g}}=\frac{f\brak{x}}{g\brak{x}}$
		\begin{enumerate}
				\begin{multicols}{2}
					\item $0\le x \le 1$
				\columnbreak
					\item $0 \le x < 1$
				\end{multicols}
				\begin{multicols}{2}
					\item $0 < x < 1$
				\columnbreak
					\item $0<x\le 1$
				\end{multicols}
		\end{enumerate}
	\item If \begin{align*}
			f\brak{x}=
		\begin{cases}
			\frac{1}{\abs{x}} &; \abs{x}\ge 1\\
			ax^2+b &;\abs{x}<1
		\end{cases}
	\end{align*}
		is differentiable at every point of the domain, then the values of $a$ and $b$ are respectively
		\begin{enumerate}
				\begin{multicols}{2}
				\item $\frac{1}{2},\frac{1}{2}$
				\columnbreak
			\item $\frac{1}{2},=\frac{3}{2}$
				\end{multicols}
				\begin{multicols}{2}
				\item $\frac{5}{2},-\frac{3}{2}$
				\columnbreak
			\item $-\frac{1}{2},\frac{3}{2}$
				\end{multicols}
		\end{enumerate}
	\item The sum pf all the $4-$digit distinct numbers that can be formed with the digits $1,2,2,3$ is,
		\begin{enumerate}
				\begin{multicols}{2}
				\item $26664$
				\columnbreak
			\item $122664$
				\end{multicols}
				\begin{multicols}{2}
				\item $122234$
				\columnbreak
			\item $22264$
				\end{multicols}
		\end{enumerate}
	\item Let ,
		\begin{align*}
			&A+2B=\myvec{1&2&0\\6&-3&3\\-5&3&1}\\
			&2A-B=\myvec{2&-1&5\\2&-1&6\\0&1&2}
		\end{align*}
		If $tr\brak{A}$ denotes the sum of all diagonal entries of the matrix $A$, then $tr\brak{A}-tr\brak{B}$ is,
		\begin{enumerate}
				\begin{multicols}{2}
					\item $0$
				\columnbreak
					\item $1$
				\end{multicols}
				\begin{multicols}{2}
					\item $2$
				\columnbreak
					\item $3$
				\end{multicols}
		\end{enumerate}
	\item The value of 
		\begin{align*}
			3+\frac{1}{4+\frac{1}{3+\frac{1}{4+\frac{1}{3+\dots \infty}}}}
		\end{align*}
		is equal to,
		\begin{enumerate}
				\begin{multicols}{2}
				\item $1.5+\sqrt{3}$
				\columnbreak
			\item $2+\sqrt{3}$
				\end{multicols}
				\begin{multicols}{2}
				\item $3+2\sqrt{3}$
				\columnbreak
			\item $4+\sqrt{3}$
				\end{multicols}
			\end{enumerate}
	\item The number of times digit $3$ will be written when listing the integers from $1$ to $1000$ is \rule{2cm}{0.1pt}
	\item The equation of the planes parallel to the plane $x-2y+2z-3=0$ which are at unit distace from the point $\brak{1,2,3}$ is $ax+by+cz+d=0$. If $\brak{b-d}=k\brak{c-a}$, then the positive value of $k$ is\rule{2cm}{0.1pt}
	\item Let $f\brak{x},g\brak{x}$ be two functions satisfying $f\brak{x^2}+g\brak{4-x}=4x^3$ and $g\brak{4-x}+g\brak{x}=0$, then the value of $\int_{-4}^4 f\brak{x^2} dx$ is,\rule{2cm}{0.1pt}
	\item The mean age of $25$ teachers in a school is $40$ years. A teacher retires at the age of $60$ years and a new teacher is appointed in his place. If the mean age of the teachers in this school now is $39$ years, then the age of the newly appointed teacher is\rule{2cm}{0.1pt}
	\item A square $ABCD$ has all its vertices on the curve $x^2y^2 = 1$. The midpoints of its sides also lie on the same curve. Then, the square of the area of $ABCD$ is\rule{2cm}{0.1pt}
	\item The missing value in the following figure is,\rule{2cm}{0.1pt}
		\begin{figure}[H]
			\centering
			\includegraphics[scale=1]{figs26/fig26.png}
			\label{stemplot}
		\end{figure}
	\item The number of solutions of the equation $\abs{\cot x}=\cot x +\brak{\frac{1}{\sin x}}$ in the interval $\sbrak{0, 2\pi}$ is\rule{2cm}{0.1pt}
	\item Let $z_1,z_2$ be the roots of the equations $z_2+a_z+12=0$ and $z_1,z_2$ form an equilateral triangle with origin. Then, the value of $\abs{a}$ is \rule{2cm}{0.1pt}
	\item Let the plane $ax+by+cz+d=0$ bisect the line joining the points $\myvec{4\\-3\\1},\myvec{2\\3\\-5}$ at right angles. If $a,b,c,d$ are integers, then the minimum value of $\brak{a^2+b^2+c^2+d^2}$ is,\rule{2cm}{0.1pt}
	\item If $f\brak{x}=\int\frac{\sbrak{5x^8+x^6}}{\sbrak{x^2+1+2x^7}^2}dx, \brak{x\ge0}$, $f\brak{0}=0$ and $f\brak{1}=\frac{1}{k}$, then the value of $k$ is, \rule{2cm}{0.1pt}
\end{enumerate}
\end{document}
