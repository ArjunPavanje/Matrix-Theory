\let\negmedspace\undefined
\let\negthickspace\undefined
\documentclass[journal]{IEEEtran}
\usepackage[a5paper, margin=10mm, onecolumn]{geometry}
\usepackage{lmodern} % Ensure lmodern is loaded for pdflatex
\usepackage{tfrupee} % Include tfrupee package

\setlength{\headheight}{1cm} % Set the height of the header box
\setlength{\headsep}{0mm}     % Set the distance between the header box and the top of the text
\usepackage{float}
\usepackage{gvv-book}
\usepackage{gvv}
\usepackage{cite}
\usepackage{amsmath,amssymb,amsfonts,amsthm}
\usepackage{algorithmic}
\usepackage{graphicx}
\usepackage{textcomp}
\usepackage{xcolor}
\usepackage{txfonts}
\usepackage{listings}
\usepackage{enumitem}
\usepackage{mathtools}
\usepackage{gensymb}
\usepackage{comment}
\usepackage[breaklinks=true]{hyperref}
\usepackage{tkz-euclide} 
\usepackage{listings}
\usepackage{gvv}                                        
\def\inputGnumericTable{}                                 
\usepackage[latin1]{inputenc}                                
\usepackage{color}                                            
\usepackage{array}                                            
\usepackage{longtable}                                       
\usepackage{calc}                                             
\usepackage{multirow}                                         
\usepackage{hhline}                                           
\usepackage{ifthen}                                           
\usepackage{lscape}
\begin{document}

\bibliographystyle{IEEEtran}
\vspace{3cm}
	
\title{2023 April 6 Shift 1}
\author{EE24BTECH11005 - Arjun Pavanje}
% \maketitle
% \newpage
% \bigskip
{\let\newpage\relax\maketitle}
\begin{enumerate}
	\setcounter{enumi}{15}
\item If $2x^y+3y^x=20$ then $\frac{dy}{dx}$ at $\myvec{2\\2}$ is equal to,
		\begin{enumerate}
				\begin{multicols}{2}
				\item $-\brak{\frac{3+\log_e8}{2+\log_e4}}$
				\columnbreak
			\item $-\brak{\frac{2+\log_e8}{3+\log_e4}}$
				\end{multicols}
				\begin{multicols}{2}
				\item $-\brak{\frac{3+\log_e4}{2+\log_e8}}$
				\columnbreak
			\item $-\brak{\frac{3+\log_e16}{4+\log_e8}}$
				\end{multicols}
		\end{enumerate}
	\item If the system of equations
		\begin{align*}
			&x+y+az=b\\
			&2x+5y+2z=6\\
			&x+2y+3z=3
		\end{align*}
		has infinitely many solutions, then $2a+3b$ is equal to, 
		\begin{enumerate}
				\begin{multicols}{2}
					\item $28$
				\columnbreak
					\item $20$
				\end{multicols}
				\begin{multicols}{2}
					\item $25$
				\columnbreak
					\item $23$
				\end{multicols}
		\end{enumerate}
	\item Let $\brak{1+x+2x^2}^{20}=a_0+a_1x+a_2x^2+\dots+a_{40}x^{40}$. Then, $a_1+a_3+a_5+\dots+a_{37}$is equal to,
		\begin{enumerate}
				\begin{multicols}{2}
				\item $2^{20}\brak{2^{20}+21}$
				\columnbreak
			\item $2^{19}\brak{2^{20}+21}$
				\end{multicols}
				\begin{multicols}{2}
				\item $2^{20}\brak{2^{20}-21}$
				\columnbreak
			\item $2^{19}\brak{2^{20}-21}$
				\end{multicols}
		\end{enumerate}
	\item Let $5f\brak{x}+4f\brak{\frac{1}{x}}=\frac{1}{x}+3, x>0$, then $\int_1^2f\brak{x}dx$ is equal to,
			\begin{enumerate}
				\begin{multicols}{2}
					\item $10\log_e2-6$
				\columnbreak
					\item $10\log_e2+6$
				\end{multicols}
				\begin{multicols}{2}
					\item $5\log_e-3$
				\columnbreak
					\item $5\log_e2+3$
				\end{multicols}
		\end{enumerate}
	\item The mean and variance of a set of $15$ numbers are $12$ and $14$ respectively. The mean and variance of another set of $15$ numbers are $14$ and $\sigma^2$ respectively. If the variance of all the $30$ numbers in the two sets is $13$, then $\sigma^2$ is equal to,
		\begin{enumerate}
				\begin{multicols}{2}
				\item $12$
				\columnbreak
			\item $10$
				\end{multicols}
				\begin{multicols}{2}
				\item $11$
				\columnbreak
			\item $9$
				\end{multicols}
			\end{enumerate}
		\item Let the tangents to the curve $x^2+2x-4y+9=0$ at the point $\vec{P}\myvec{1\\3}$ on it meet the y-axis at $\vec{A}$. Let the line passing through $\vec{P}$ and parallel to the line $x-3y=6$ meet the parabola $y^2=4x$ at $\vec{B}$. If $\vec{B}$ lies on the line $2x-3y=8$, then $\brak{AB}^2$ us equal to \rule{2cm}{0.1pt}
		\item Let the point $\myvec{p\\p+1}$ lie inside the region 
			\begin{align*}
				E=\cbrak{\brak{x,y}: 3-x \le y \le \sqrt{9-x^2}, 0 \le x \le 3}
			\end{align*}
		If the set of all values of $p$ in the interval $\myvec{a\\b}$ then $b^2+b-a^2$ is equal to \rule{2cm}{0.1pt}
	\item Let $y=y\brak{x}$ be a solution of the differential equation 
		\begin{align*}
			\brak{x\cos x}dy+\brak{xy\sin x +y\cos x-1}dx=0,0<x<\frac{\pi}{2}
		\end{align*}
		If $\frac{\pi}{3}y\brak{\frac{\pi}{3}}=\sqrt{3}$, then $\abs{\frac{\pi}{6}y^{''}\brak{\frac{\pi}{6}}+2y^{'}\brak{\frac{\pi}{6}}}$\rule{2cm}{0.1pt}
	\item The Let $a \in Z$ and $\sbrak{t}$ be the greatest integer $\le t$. Then the number of points, where the function $f\brak{x}=\sbrak{a+13\sin x}, x\in\brak{0,\pi}$ is not differentiable is\rule{2cm}{0.1pt}
	\item If the area of the region 
		\begin{align*}
			S=\cbrak{\brak{x,y}:2y-y^2 \le x^2 \le 2y, x \ge y}
		\end{align*}
		is equal to $\brak{\frac{n+2}{n+1}-\frac{\pi}{n-1}}$ then the natural number $n$ is equal to \rule{2cm}{0.1pt}
	\item The number of ways of giving $20$ distinct oranges to $3$ children such that each child gets atleast one orange is \rule{2cm}{0.1pt}
	\item Let the image of the point $\vec{P}\myvec{1\\2\\3}$ in the plane $2x-y+z=9$ be $\vec{Q}$. If the coordinates of the point $\vec{R}$ are $\myvec{6\\10\\7}$. Then the square of the area of triangle $PQR$ is \rule{2cm}{0.1pt}
	\item Let A circle passing through the point $\vec{P}\myvec{\alpha\\\beta}$ in the first quadrant touches the two coordinate axes at the points $\vec{A},\vec{B}$. The point $\vec{P}$ is above the line $\vec{AB}$. The point $\vec{Q}$ on the line segment $\vec{AB}$ is the foot of perpendicular from $\vec{P}$ on $\vec{AB}$. If $\vec{PQ}$ is equal to $11$ units, then value of $\alpha\beta$ is \rule{2cm}{0.1pt}
	\item The coeffecient of $x^{18}$ in the expansion of $\brak{x^4-\frac{1}{x^3}}^{15}$is \rule{2cm}{0.1pt}
	\item Let $A=\cbrak{1,2,3,4,\dots,10}, B=\cbrak{0,1,2,3,4}$. The number of elements in the relation $R=\cbrak{\brak{a,b}\in A \times A: 2\brak{a-b}^2+3\brak{a-b} \in B}$ is \rule{2cm}{0.1pt}
\end{enumerate}
\end{document}
