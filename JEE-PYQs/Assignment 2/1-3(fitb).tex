\iffalse
\let\negmedspace\undefined
\let\negthickspace\undefined
\documentclass[journal,12pt,onecolumn,article]{IEEEtran}
\usepackage{multicol}
\usepackage{cite}
\usepackage{amsmath,amssymb,amsfonts,amsthm}
\usepackage{algorithmic}
\usepackage{graphicx}
\usepackage{textcomp}
\usepackage{xcolor}
\usepackage{txfonts}
\usepackage{listings}
\usepackage{enumitem}
\usepackage{mathtools}
\usepackage{gensymb}
\usepackage{comment}
\usepackage[breaklinks=true]{hyperref}
\usepackage{tkz-euclide} 
\usepackage{listings}
\usepackage{gvv}                                        
%\def\inputGnumericTable{}                                 
\usepackage[latin1]{inputenc}                                
\usepackage{color}                                            
\usepackage{array}                                            
\usepackage{longtable}                                       
\usepackage{calc}                                             
\usepackage{multirow}                                         
\usepackage{hhline}                                           
\usepackage{ifthen}                                           
\usepackage{lscape}
\usepackage{tabularx}
\usepackage{array}
\usepackage{float}


\newtheorem{theorem}{Theorem}[section]
\newtheorem{problem}{Problem}
\newtheorem{proposition}{Proposition}[section]
\newtheorem{lemma}{Lemma}[section]
\newtheorem{corollary}[theorem]{Corollary}
\newtheorem{example}{Example}[section]
\newtheorem{definition}[problem]{Definition}
\newcommand{\BEQA}{\begin{eqnarray}}
\newcommand{\EEQA}{\end{eqnarray}}
\newcommand{\define}{\stackrel{\triangle}{=}}
\theoremstyle{remark}
\newtheorem{rem}{Remark}

% Marks the beginning of the document
\begin{document}
\bibliographystyle{IEEEtran}
\vspace{3cm}

\title{Assignment 2}
\author{EE24BTECH11005 - Arjun Pavanje}
\maketitle
\bigskip
\renewcommand{\thefigure}{\theenumi}
\renewcommand{\thetable}{\theenumi}
\section{fitb}
\fi
%\begin{enumerate}
\item Let $a,b,c$ be positive real numbers. Let
\begin{multline*}
\theta=\tan ^{-1}\brak{\sqrt{\frac{a\brak{a+b+c}}{bc}}} + \tan ^{-1}\brak{\sqrt{\frac{b\brak{a+b+c}}{ca}}}\ + \tan ^{-1}\brak{\sqrt{\frac{c\brak{a+b+c}}{ab}}} 
\end{multline*}
Then $\tan \brak{\theta}= \rule{2cm}{0.1pt}$ 
\hfill \brak{1981-2Marks}
\item The numerical value of $\tan \cbrak{ 2\tan ^{-1}\brak{\frac{1}{5}}-\frac{\pi}{4}}$ is equal to \rule{2cm}{0.1pt}
\hfill \brak{1984-2Marks}
\item The greater of the two angles 
\begin{align*}
A &= 2 \tan ^{-1}\brak{2\sqrt{2}-1} \text{ and}\\
B &= 3\sin ^{-1}\brak{\frac{1}{3}} + \sin ^{-1}\brak{\frac{3}{5}}
\end{align*}
is \rule{2cm}{0.1pt}
\hfill \brak{1989-2Marks}
%\end{enumerate}

