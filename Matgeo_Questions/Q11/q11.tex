\let\negmedspace\undefined
\let\negthickspace\undefined
\documentclass[journal]{IEEEtran}
\usepackage[a5paper, margin=10mm, onecolumn]{geometry}
\usepackage{lmodern} % Ensure lmodern is loaded for pdflatex
\usepackage{tfrupee} % Include tfrupee package

\setlength{\headheight}{1cm} % Set the height of the header box
\setlength{\headsep}{0mm}     % Set the distance between the header box and the top of the text

\usepackage{gvv-book}
\usepackage{gvv}
\usepackage{cite}
\usepackage{amsmath,amssymb,amsfonts,amsthm}
\usepackage{algorithmic}
\usepackage{graphicx}
\usepackage{textcomp}
\usepackage{xcolor}
\usepackage{txfonts}
\usepackage{listings}
\usepackage{enumitem}
\usepackage{mathtools}
\usepackage{gensymb}
\usepackage{comment}
\usepackage[breaklinks=true]{hyperref}
\usepackage{tkz-euclide} 
\usepackage{listings}
\usepackage{gvv}                                        
\def\inputGnumericTable{}                                 
\usepackage[latin1]{inputenc}                                
\usepackage{color}                                            
\usepackage{array}                                            
\usepackage{longtable}                                       
\usepackage{calc}                                             
\usepackage{multirow}                                         
\usepackage{hhline}                                           
\usepackage{ifthen}                                           
\usepackage{lscape}
\begin{document}

\bibliographystyle{IEEEtran}
\vspace{3cm}

\title{3-3.2-28}
\author{EE24BTECH11005 - Arjun Pavanje}
% \maketitle
% \newpage
% \bigskip
{\let\newpage\relax\maketitle}
Question:\\
A triangle $ABC$ can be constructed in which $AB =5cm,\angle{A=}45^{\degree}$ and $BC+AC=5cm$
\begin{table}[h!]    
  \centering
  \begin{tabular}[12pt]{ |c| c|}
    \hline
    \textbf{Variable} & \textbf{Description}\\ 
    \hline
	$2x+3y=9$ & given line\\
   \hline

    \end{tabular}

  \caption{Variables Used}
  \label{tab1-1.9-6}
\end{table}\\
\solution
Using cosine formula in $\Delta ABC$,
\begin{align}
	a^2&=b^2+c^2-2bc\cos{A}\\
	\brak{K-b}^2&=b^2+c^2-2bc\cos{A}\\
	b&=\frac{K^2-c^2}{2\brak{K-c\cos{A}}}
\end{align}
Then the coordinates of $\Delta ABC$ can be represented as
\begin{align}
	\vec{A}=\myvec{0\\0}, \vec{B}=\myvec{c\\0}, \vec{C}=b\myvec{\cos{A}\\\sin{A}}
\end{align}
Substituting values, we get
\begin{align}
	b&=0\\
	\therefore \vec{A}&=\myvec{0\\0},\vec{B}=\myvec{5\\0}, \vec{C}=\myvec{0\\0}
\end{align}
As two of the points $\brak{\vec{A},\vec{C}}$ coincide, triangle of given dimensions cannot be constructed.
\begin{figure}[h!]
   \centering
   \includegraphics[width = 1\linewidth]{figs/fig.png}
   \caption{Plot of the triangle}
   \label{stemplot}
\end{figure}
\end{document}
