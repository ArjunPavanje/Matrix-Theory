\let\negmedspace\undefined
\let\negthickspace\undefined
\documentclass[journal]{IEEEtran}
\usepackage[a5paper, margin=10mm, onecolumn]{geometry}
\usepackage{lmodern} % Ensure lmodern is loaded for pdflatex
\usepackage{tfrupee} % Include tfrupee package

\setlength{\headheight}{1cm} % Set the height of the header box
\setlength{\headsep}{0mm}     % Set the distance between the header box and the top of the text

\usepackage{gvv-book}
\usepackage{gvv}
\usepackage{cite}
\usepackage{amsmath,amssymb,amsfonts,amsthm}
\usepackage{algorithmic}
\usepackage{graphicx}
\usepackage{textcomp}
\usepackage{xcolor}
\usepackage{txfonts}
\usepackage{listings}
\usepackage{enumitem}
\usepackage{mathtools}
\usepackage{gensymb}
\usepackage{comment}
\usepackage[breaklinks=true]{hyperref}
\usepackage{tkz-euclide} 
\usepackage{listings}
\usepackage{gvv}                                        
\def\inputGnumericTable{}                                 
\usepackage[latin1]{inputenc}                                
\usepackage{color}                                            
\usepackage{array}                                            
\usepackage{longtable}                                       
\usepackage{calc}                                             
\usepackage{multirow}                                         
\usepackage{hhline}                                           
\usepackage{ifthen}                                           
\usepackage{lscape}
\begin{document}

\bibliographystyle{IEEEtran}
\vspace{3cm}

\title{1-1.9-6}
\author{EE24BTECH11005 - Arjun Pavanje
}
% \maketitle
% \newpage
% \bigskip
{\let\newpage\relax\maketitle}
Question:\\
If $\vec{Q}=\myvec{0\\1}$ is equidistant from $\vec{P}=\myvec{5\\-3}$ and $\vec{R}=\myvec{x\\6}$, find the value of $x$.\\
\solution
\begin{table}[h!]    
  \centering
  \begin{tabular}[12pt]{ |c| c|}
    \hline
    \textbf{Variable} & \textbf{Description}\\ 
    \hline
	$2x+3y=9$ & given line\\
   \hline

    \end{tabular}

  \caption{Variables Used}
  \label{tab1-1.9-6}
\end{table}\\
As, $\vec{Q}$ is equidistant from $\vec{P}$, $\vec{R}$\\
\begin{align}
	\norm{Q-P} &= \norm{Q-R}\\
		\sqrt{\brak{Q-P}^T\brak{Q-P}}&=\sqrt{\brak{Q-R}^T\brak{Q-R}}
\end{align}
$$		\brak{Q-P} = \myvec{
			-5\\
			4
		},
		\brak{Q-R}=\myvec{
			-x\\
			-5
		}\\
$$
Putting values into equation (2) and squaring,
\begin{align}
	25+16&=x^2+25\\
	x^2&=16\\
	x &= \pm{4}
\end{align}
The required values of $x$ are $+4,-4$
\begin{figure}[h!]
   \centering
   \includegraphics[width = 1\linewidth]{figs/fig.png}
   \caption{Plot of P,Q,R}
   \label{stemplot}
\end{figure}
\end{document}
