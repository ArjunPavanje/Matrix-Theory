\let\negmedspace\undefined
\let\negthickspace\undefined
\documentclass[journal]{IEEEtran}
\usepackage[a5paper, margin=10mm, onecolumn]{geometry}
\usepackage{lmodern} % Ensure lmodern is loaded for pdflatex
\usepackage{tfrupee} % Include tfrupee package

\setlength{\headheight}{1cm} % Set the height of the header box
\setlength{\headsep}{0mm}     % Set the distance between the header box and the top of the text

\usepackage{gvv-book}
\usepackage{gvv}
\usepackage{cite}
\usepackage{amsmath,amssymb,amsfonts,amsthm}
\usepackage{algorithmic}
\usepackage{graphicx}
\usepackage{textcomp}
\usepackage{xcolor}
\usepackage{txfonts}
\usepackage{listings}
\usepackage{enumitem}
\usepackage{mathtools}
\usepackage{gensymb}
\usepackage{comment}
\usepackage[breaklinks=true]{hyperref}
\usepackage{tkz-euclide} 
\usepackage{listings}
\usepackage{gvv}                                        
\def\inputGnumericTable{}                                 
\usepackage[latin1]{inputenc}                                
\usepackage{color}                                            
\usepackage{array}                                            
\usepackage{longtable}                                       
\usepackage{calc}                                             
\usepackage{multirow}                                         
\usepackage{hhline}                                           
\usepackage{ifthen}                                           
\usepackage{lscape}
\begin{document}

\bibliographystyle{IEEEtran}
\vspace{3cm}

\title{1-1.11-9}
\author{EE24BTECH11005 - Arjun Pavanje
}
% \maketitle
% \newpage
% \bigskip
{\let\newpage\relax\maketitle}
Question:\\
Find a unit vector perpendicular to both vectors $\vec{a}\myvec{1\\-7\\7}$ and $\vec{b}\myvec{3\\-2\\2}$
\begin{table}[h!]    
  \centering
  \begin{tabular}[12pt]{ |c| c|}
    \hline
    \textbf{Variable} & \textbf{Description}\\ 
    \hline
	$2x+3y=9$ & given line\\
   \hline

    \end{tabular}

  \caption{Variables Used}
  \label{tab1-1.9-6}
\end{table}\\
\solution
Let $\vec{x}$ be a vector perpendicular to both $\vec{a},\vec{b}$. Then,
\begin{align}
	\myvec{a^T\\b^T}\vec{x}=0\\
	\myvec{1&-7&7\\3&-2&2}\vec{x}=0
\end{align}
Using row reduction, then back substitution,
\begin{align}
	\myvec{1&-7&7\\3&-2&2}&\xleftrightarrow[]{R_2=R_2+R_1}\myvec{1&-7&7\\4&0&0}\\
	x_1=0, x_2=x_3\\
	\vec{x}=x_2\myvec{0\\1\\1}
\end{align}
Here, $x_2 \in R$, we need to pick it such that magnitude is $1$
\begin{align}
	\norm{x}&=\sqrt{x^Tx}=\sqrt{(x_2)^2\myvec{0&1&1}\myvec{0\\1\\1}}=1\\
	\sqrt{2{x_2}^2}&=1\\
	x_2&=\pm \frac{1}{\sqrt{2}}
\end{align}
Required unit vector is $\frac{1}{\sqrt{2}}\myvec{0\\1\\1}$ or$\frac{-1}{\sqrt{2}}\myvec{0\\1\\1}$ 
\begin{figure}[h!]
   \centering
   \includegraphics[width = 1\linewidth]{figs/fig.png}
   \caption{Plot of the vectors}
   \label{stemplot}
\end{figure}
\end{document}
