\let\negmedspace\undefined
\let\negthickspace\undefined
\documentclass[journal]{IEEEtran}
\usepackage[a5paper, margin=10mm, onecolumn]{geometry}
\usepackage{lmodern} % Ensure lmodern is loaded for pdflatex
\usepackage{tfrupee} % Include tfrupee package

\setlength{\headheight}{1cm} % Set the height of the header box
\setlength{\headsep}{0mm}     % Set the distance between the header box and the top of the text

\usepackage{gvv-book}
\usepackage{gvv}
\usepackage{cite}
\usepackage{amsmath,amssymb,amsfonts,amsthm}
\usepackage{algorithmic}
\usepackage{graphicx}
\usepackage{textcomp}
\usepackage{xcolor}
\usepackage{txfonts}
\usepackage{listings}
\usepackage{enumitem}
\usepackage{mathtools}
\usepackage{gensymb}
\usepackage{comment}
\usepackage[breaklinks=true]{hyperref}
\usepackage{tkz-euclide} 
\usepackage{listings}
\usepackage{gvv}                                        
\def\inputGnumericTable{}                                 
\usepackage[latin1]{inputenc}                                
\usepackage{color}                                            
\usepackage{array}                                            
\usepackage{longtable}                                       
\usepackage{calc}                                             
\usepackage{multirow}                                         
\usepackage{hhline}                                           
\usepackage{ifthen}                                           
\usepackage{lscape}
\begin{document}

\bibliographystyle{IEEEtran}
\vspace{3cm}

\title{9-9.3-10}
\author{EE24BTECH11005 - Arjun Pavanje}
% \maketitle
% \newpage
% \bigskip
{\let\newpage\relax\maketitle}
Question:\\
If the area of the region bounded by the curve $y^2=4ax$ and the line $x=4a$ is $\frac{256}{3}$ sq. units, then find the positive value of $a$, where $a>0$
\begin{table}[h!]    
  \centering
  \begin{tabular}[12pt]{ |c| c|}
    \hline
    \textbf{Variable} & \textbf{Description}\\ 
    \hline
	$2x+3y=9$ & given line\\
   \hline

    \end{tabular}

  \caption{Variables Used}
  \label{tab1-1.9-6}
\end{table}\\
\solution
Line equation of form $\vec{x}=\vec{h}+k\vec{m}$
\begin{align}
\vec{x}=\myvec{4a\\0}+k\myvec{0\\1}
\end{align}
Equation of parabola of form $\vec{x}^{\top}\vec{V}\vec{x}+2\vec{u}^{\top}\vec{x}+f=0$ is
\begin{align}
	\vec{u}=\myvec{-2a\\0},f=0,\vec{V}=\myvec{0&0\\0&1}
\end{align}
If a line intersects the conic, $k$ value of intersecting point is given by,
\begin{align}
	k_i=\frac{-\vec{m}^{\top}\brak{\vec{Vh}+\vec{u}}\pm\sqrt{\sbrak{\vec{m}^{\top}\brak{\vec{Vh}+\vec{u}}}^2-g\brak{h}\brak{\vec{m}^{\top}\vec{Vm}}}}{\vec{m}^{\top}\vec{Vm}}
\end{align}
On substituting values of $\vec{u},\vec{m},\vec{h},\vec{V}$ we get,
\begin{align}
	k=\pm4a
\end{align}
Points of intersection with parabola are, $\myvec{4a\\4a},\myvec{4a\\-4a}$
Area bound between the parabola and the line is,
\begin{align}
	2\int_0^{4a} 2\sqrt{a}\sqrt{x}dx\\
	=\frac{64a^2}{3}=\frac{256}{3}
\end{align}
We get $a=2$\\
Required value of $a$ is $2$
\begin{figure}[h!]
   \centering
   \includegraphics[width = 1\linewidth]{figs/fig.png}
   \caption{Parabola $y^2=8x$, Line $x=8$}
   \label{stemplot}
\end{figure}
\end{document}
