%iffalse
\let\negmedspace\undefined
\let\negthickspace\undefined
\documentclass[journal,12pt,twocolumn]{IEEEtran}
\usepackage{cite}
\usepackage{amsmath,amssymb,amsfonts,amsthm}
\usepackage{algorithmic}
\usepackage{graphicx}
\usepackage{textcomp}
\usepackage{xcolor}
\usepackage{txfonts}
\usepackage{listings}
\usepackage{enumitem}
\usepackage{mathtools}
\usepackage{gensymb}
\usepackage{comment}
\usepackage[breaklinks=true]{hyperref}
\usepackage{tkz-euclide} 
\usepackage{listings}
\usepackage{gvv}                                        
%\def\inputGnumericTable{}                                 
\usepackage[latin1]{inputenc}                                
\usepackage{color}                                            
\usepackage{array}                                            
\usepackage{longtable}                                       
\usepackage{calc}                                             
\usepackage{multirow}                                         
\usepackage{hhline}                                           
\usepackage{ifthen}                                           
\usepackage{lscape}
\usepackage{tabularx}
\usepackage{array}
\usepackage{float}
\usepackage{tikz}
\usepackage{circuitikz}

\newtheorem{theorem}{Theorem}[section]
\newtheorem{problem}{Problem}
\newtheorem{proposition}{Proposition}[section]
\newtheorem{lemma}{Lemma}[section]
\newtheorem{corollary}[theorem]{Corollary}
\newtheorem{example}{Example}[section]
\newtheorem{definition}[problem]{Definition}
\newcommand{\BEQA}{\begin{eqnarray}}
\newcommand{\EEQA}{\end{eqnarray}}
\newcommand{\define}{\stackrel{\triangle}{=}}
\theoremstyle{remark}
\newtheorem{rem}{Remark}

% Marks the beginning of the document
\begin{document}
\bibliographystyle{IEEEtran}
\renewcommand{\thefigure}{\theenumi}
\renewcommand{\thetable}{\theenumi}
\begin{circuitikz}
   
    \draw (0,0)--(0,4) node[label={[font=\footnotesize]left:A},color=cyan]{}
    to[resistor,i=$ $,l=$10 \Omega$, color=cyan,*-*] (2,6) node[label={[font=\footnotesize]above:B}]{}
    to[resistor,i= $ $,l=$5 \Omega$,color=cyan,*-*] (4,4) node[label={[font=\footnotesize]above:C}]{}
    ;
    \draw (0,4) to[resistor,i=$ $,l_=$5 \Omega$,color=cyan,*-*] (2,2) node[label={[font=\footnotesize]below:D}]{}
    to[resistor,i=$ $,l_=$10 \Omega$,color=cyan,*-*] (4,4)    
    ;
    \draw (4,4)--(4,0) 
    ;
    \draw (2,0) to [resistor,i=$ $,l=$10\Omega$,color=cyan] (0,0)
    ;
	\draw (2,0) to[battery1,l_=$10V$,color=cyan] (4,0);
    \draw (2,6) to[resistor, i=$ $,l=$5\Omega$,color=cyan] (2,2)
    ;
\end{circuitikz}

\end{document}
