\let\negmedspace\undefined
\let\negthickspace\undefined
\documentclass[journal,12pt,article,twocolumn]{IEEEtran}
\usepackage{cite}
\usepackage{amsmath,amssymb,amsfonts,amsthm}
\usepackage{algorithmic}
\usepackage{graphicx}
\usepackage{textcomp}
\usepackage{xcolor}
\usepackage{txfonts}
\usepackage{listings}
\usepackage{enumitem}
\usepackage{mathtools}
\usepackage{gensymb}
\usepackage{comment}
\usepackage[breaklinks=true]{hyperref}
\usepackage{tkz-euclide} 
\usepackage{listings}
\usepackage{gvv}                                        
\def\inputGnumericTable{}                                 
\usepackage[latin1]{inputenc}                                
\usepackage{color}                                            
\usepackage{array}                                            
\usepackage{longtable}                                       
\usepackage{calc}                                             
\usepackage{multirow}                                         
\usepackage{hhline}                                           
\usepackage{ifthen}                                           
\usepackage{lscape}

\newtheorem{theorem}{Theorem}[section]
\newtheorem{problem}{Problem}
\newtheorem{proposition}{Proposition}[section]
\newtheorem{lemma}{Lemma}[section]
\newtheorem{corollary}[theorem]{Corollary}
\newtheorem{example}{Example}[section]
\newtheorem{definition}[problem]{Definition}
\newcommand{\BEQA}{\begin{eqnarray}}
\newcommand{\EEQA}{\end{eqnarray}}
\newcommand{\define}{\stackrel{\triangle}{=}}
\theoremstyle{remark}
\newtheorem{rem}{Remark}
\begin{document}

\bibliographystyle{IEEEtran}
\vspace{3cm}

\title{ASSIGNMENT 1}
\author{EE24BTECH11005 - Arjun Pavanje$^{*}$% <-this % stops a space
}
\maketitle
\newpage
\bigskip
\begin{enumerate}
\item Without using tables prove that 
\begin{align*} 
\brak{\sin\brak{12^{\degree}}}\brak{\sin\brak{48^{\degree}}}\brak{\sin\brak{54^{\degree}}}= \frac{1}{8}
\end{align*}
\hfill \brak{1982 -2 Marks}
\item Show that 
\begin{multline*}
16\brak{\cos\brak{\frac{2\pi}{15}}}\brak{\cos\brak{\frac{4\pi}{15}}}\brak{\cos\brak{\frac{8\pi}{15}}}\\\brak{\cos\brak{\frac{16\pi}{15}}}=1
\end{multline*}
\hfill\brak{1983-2 Marks}
\item Find all the solution of 
\begin{align*}
4\cos^2\brak{x} \sin \brak{x} -2\sin^2\brak{x} = 3\sin \brak{x}
\end{align*}
\hfill\brak{1983-2 Marks}
\item Find the values of $x \in \brak{-\pi, +\pi}$ which satisfy the equation
\begin{align*}
	8^{\brak{1+\abs{\cos \brak{x}}+\abs{\cos^2\brak{x}}+\abs{\cos^3\brak{x}}+\dots}}= 4^3
\end{align*}
\hfill\brak{1984-2 Marks}
\item Prove that 
\begin{multline*}
\tan \brak{\alpha}+2\tan \brak{2\alpha}+4\tan \brak{4\alpha}+8\cot \brak{8\alpha}\\=\cot \brak{\alpha}
\end{multline*}
\hfill\brak{1988-2 Marks}
\item $ABC$ is a triangle such that 
\begin{multline*}
\sin{\brak{2A+B}}=\sin{\brak{C-A}}=-\sin{\brak{B+2C}}\\=\frac{1}{2}
\end{multline*}
If $A$, $B$ and $C$ are in arithmetic progression, determine the values of $A$, $B$ and $C$.
\hfill\brak{1990- 5 Marks}
\item If $\exp \cbrak{\brak{\sin^2\brak{x}+\sin^4\brak{x}+\sin^6\brak{x}+\dots\infty}\ln 2}$ satisfies the equation $x^2-9x+8$, find the value of $\frac{\cos \brak{x}}{\cos \brak{x} + \sin \brak{x}}$, 0$<$x$<$$\frac{\pi}{2}$
\hfill\brak{1991-4 Marks}
\item Show that the value of  $\frac{\tan \brak{x}}{\tan \brak{3x}}$, wherever defined never lies between $\frac{1}{3}$ and 3
\hfill\brak{1992-4 Marks}
\item Determine the smallest positive value of $x$ \brak{\text{in degrees}} for which 
\begin{align*}
\tan {\brak{x+100^{\degree}}}=\tan {\brak{x+50^{\degree}}}\tan \brak{x}\tan {\brak{x-50^{\degree}}}
\end{align*}
\hfill\brak{1993-5 Marks}
\item Find the smallest positive number $p$ for which the equation 
\begin{align*}
\cos {\brak{p\sin \brak{x}}}=\sin {\brak{p\cos \brak{x}}}
\end{align*}
has a solution $ x \in [0, \pi]$
\hfill\brak{1995- 5 Marks}
\item Find all values of $\theta$ in the interval $\brak{-\frac{\pi}{2},\frac{\pi}{2}}$ satisfying the equation 
\begin{align*}
\brak{1-\tan \brak{\theta}}\brak{1+\tan \brak{\theta}}\sec^2\brak{\theta}+ 2^{\tan^2\brak{\theta}}=0
\end{align*}
\hfill\brak{1996-2 Marks}
\item Prove that the values of the function 
\begin{align*}
\frac{\sin \brak{x} \cos \brak{3x}}{\sin \brak{3x} \cos \brak{x}}
\end{align*}
does not lie between $\frac{1}{3}$ and 3 for any real $x$\\
\hfill\brak{1997-5 Marks}
\item Prove that 
\begin{align*}
\sum_{k=1}^{n-1} \brak{n-k}\cos\brak{ \frac{2k\pi}{n}}=-\frac{n}{2}
\end{align*}
, where $n\ge3$
\hfill\brak{1997-5 Marks}
\item In any triangle $ABC$, prove that 
\begin{multline*}
\cot \brak{\frac{A}{2}}+\cot \brak{\frac{B}{2}}+\cot \brak{\frac{C}{2}}=\\\cot \brak{\frac{A}{2}}\cot \brak{\frac{B}{2}}\cot \brak{\frac{C}{2}}
\end{multline*}
\hfill\brak{2000-3 Marks}
\item Find the range of values of for which 
\begin{align*}
2\sin \brak{t} = \frac{1-2x+5x^2}{3x^2-2x-1}
\end{align*}
, t $\in \sbrak{-\frac{\pi}{2},\frac{\pi}{2}}$
\hfill\brak{2005-2 Marks}
\end{enumerate}
\end{document}
