\let\negmedspace\undefined
\let\negthickspace\undefined
\documentclass[journal,12pt,article,twocolumn]{IEEEtran}
\usepackage{cite}
\usepackage{amsmath,amssymb,amsfonts,amsthm}
\usepackage{algorithmic}
\usepackage{graphicx}
\usepackage{textcomp}
\usepackage{xcolor}
\usepackage{txfonts}
\usepackage{listings}
\usepackage{enumitem}
\usepackage{mathtools}
\usepackage{gensymb}
\usepackage{comment}
\usepackage[breaklinks=true]{hyperref}
\usepackage{tkz-euclide} 
\usepackage{listings}
\usepackage{gvv}                                        
\def\inputGnumericTable{}                                 
\usepackage[latin1]{inputenc}                                
\usepackage{color}                                            
\usepackage{array}                                            
\usepackage{longtable}                                       
\usepackage{calc}                                             
\usepackage{multirow}                                         
\usepackage{hhline}                                           
\usepackage{ifthen}                                           
\usepackage{lscape}

\newtheorem{theorem}{Theorem}[section]
\newtheorem{problem}{Problem}
\newtheorem{proposition}{Proposition}[section]
\newtheorem{lemma}{Lemma}[section]
\newtheorem{corollary}[theorem]{Corollary}
\newtheorem{example}{Example}[section]
\newtheorem{definition}[problem]{Definition}
\newcommand{\BEQA}{\begin{eqnarray}}
\newcommand{\EEQA}{\end{eqnarray}}
\newcommand{\define}{\stackrel{\triangle}{=}}
\theoremstyle{remark}
\newtheorem{rem}{Remark}
\begin{document}

\bibliographystyle{IEEEtran}
\vspace{3cm}

\title{ASSIGNMENT 1}
\author{EE24BTECH11005 - Arjun Pavanje$^{*}$% <-this % stops a space
}
\maketitle
\newpage
\bigskip
\section*{\color{black}\colorbox{gray}{\textbf{\large {E:}}}\color{white}\colorbox{magenta}{\textbf{\large {Subjective Questions}}}}
\begin{enumerate}
\item[{\textcolor{magenta}{6.}}] Without using tables prove that $(\sin{12^\circ} )(\sin{48^\circ})
(\sin{54^\circ})= \frac{1}{8}$
\begin{flushright}
\textcolor{magenta}{\textit{\brak{1982 -2 Marks}}} 
\end{flushright}
\item[\textcolor{magenta}{7.}] Show that $16(\cos{\frac{2\pi}{15}})(\cos{\frac{4\pi}{15}})(\cos{\frac{8\pi}{15}})(\cos{\frac{16\pi}{15}})=1$
\begin{flushright}
  \textcolor{magenta}{\textit{\brak{1983-2 Marks}}} 
\end{flushright}
\item[\textcolor{magenta}{8.}] Find all the solution of $4\cos^2x \sin x -2\sin^2x = 3\sin x$
\begin{flushright}
    \textcolor{magenta}{\textit{(1983-2 Marks)}} 
\end{flushright}
\item[\textcolor{magenta}{9.}] Find the values of \( x \in (-\pi, +\pi) \) which satisfy the equation
$8^{(1+|\cos x|+|\cos^2x|+|\cos^3x|+\dots)}= 4^3$
\begin{flushright}
   \textcolor{magenta}{\textit{\brak{1984-2 Marks}}}  
\end{flushright}
\item[\textcolor{magenta}{10.}] Prove that $\tan \alpha+2\tan 2\alpha+4\tan 4\alpha+8\cot 8\alpha=\cot \alpha$
\begin{flushright}
    \textcolor{magenta}{\textit{\brak{1988-2 Marks}}} 
\end{flushright}
\item[\textcolor{magenta}{11.}] ABC is a triangle such that $\sin{(2A+B)}=\sin{(C-A)}=-\sin{(B+2C)}=\frac{1}{2}$ If A, B and C are in arithmetic progression, determine the values of A, B and C. 
\begin{flushright}
    \textcolor{magenta}{\textit{\brak{1990- 5 Marks}}}
\end{flushright}
\item[\textcolor{magenta}{12.}] If $exp \{(\sin^2x+\sin^4x+\sin^6x+\dots\infty)ln2\}$ satisfies the equation $x^2-9x+8$, find the value of $\frac{\cos x}{\cos x + \sin x}$, 0$<$x$<$$\frac{\pi}{2}$
\begin{flushright}
    \textcolor{magenta}{\textit{\brak{1991-4 Marks}}} 
\end{flushright}
\item[\textcolor{magenta}{13.}] Show that the value of  $\frac{\tan x}{\tan 3x}$, wherever defined never lies between $\frac{1}{3}$ and 3
\begin{flushright}
    \textcolor{magenta}{\textit{\brak{1992-4 Marks}}}
\end{flushright}
\item[\textcolor{magenta}{14.}] Determine the smallest positive value of $x$ (in degrees) for which $\tan (x+100^\circ)=\tan (x+50^\circ)\tan x\tan (x-50^\circ)$
\begin{flushright}
   \textcolor{magenta}{\textit{\brak{1993-5 Marks}}} 
\end{flushright}
\item[\textcolor{magenta}{15.}] Find the smallest positive number $p$ for which the equation $\cos (p\sin x)=\sin (p\cos x)$ has a solution \(x \in [0, \pi]\)
\begin{flushright}
    \textcolor{magenta}{\textit{\brak{1995- 5 Marks}}\textit{)}} 
\end{flushright}
\item[\textcolor{magenta}{16.}] Find all values of $\theta$ in the interval ($-\frac{\pi}{2},\frac{\pi}{2})$) satisfying the equation $(1-\tan \theta)(1+\tan \theta)\sec^2\theta+ 2^{\tan^2 \theta}=0$
\begin{flushright}
    \textcolor{magenta}{\textit{\brak{1996-2 Marks}}}
\end{flushright}
\item[\textcolor{magenta}{17.}] Prove that the values of the function $\frac{\sin x \cos 3x}{\sin 3x \cos x}$ does not lie between $\frac{1}{3}$ and 3 for any real $x$
\begin{flushright}
    \textcolor{magenta}{\textit{\brak{1997-5 Marks}}} 
\end{flushright}
\item[\textcolor{magenta}{18.}] Prove that $\sum_{k=1}^{n-1}$ $(n-k)\cos \frac{2k\pi}{n}=-\frac{n}{2}$, where n$\ge$3
\begin{flushright}
    \textcolor{magenta}{\textit{\brak{1997-5 Marks}}} 
\end{flushright}
\item[\textcolor{magenta}{19.}] In any triangle ABC, prove that $\cot \frac{A}{2}+\cot \frac{B}{2}+\cot \frac{C}{2}=\cot \frac{A}{2}\cot \frac{B}{2}\cot \frac{C}{2}$
\begin{flushright}
    \textcolor{magenta}{\textit{\brak{2000-3 Marks}}} 
\end{flushright}
\item[\textcolor{magenta}{20.}] Find the range of values oft for which $2\sin t = \frac{1-2x+5x^2}{3x^2-2x-1}$, t $\in \left[-\frac{\pi}{2},\frac{\pi}{2}\right]$
\begin{flushright}
    \textcolor{magenta}{\textit{\brak{2005-2 Marks}}}
\end{flushright}
\end{enumerate}
\end{document}
