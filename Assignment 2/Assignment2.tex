%iffalse
\let\negmedspace\undefined
\let\negthickspace\undefined
\documentclass[journal,12pt,twocolumn,article]{IEEEtran}
\usepackage{multicol}
\usepackage{cite}
\usepackage{amsmath,amssymb,amsfonts,amsthm}
\usepackage{algorithmic}
\usepackage{graphicx}
\usepackage{textcomp}
\usepackage{xcolor}
\usepackage{txfonts}
\usepackage{listings}
\usepackage{enumitem}
\usepackage{mathtools}
\usepackage{gensymb}
\usepackage{comment}
\usepackage[breaklinks=true]{hyperref}
\usepackage{tkz-euclide} 
\usepackage{listings}
\usepackage{gvv}                                        
%\def\inputGnumericTable{}                                 
\usepackage[latin1]{inputenc}                                
\usepackage{color}                                            
\usepackage{array}                                            
\usepackage{longtable}                                       
\usepackage{calc}                                             
\usepackage{multirow}                                         
\usepackage{hhline}                                           
\usepackage{ifthen}                                           
\usepackage{lscape}
\usepackage{tabularx}
\usepackage{array}
\usepackage{float}


\newtheorem{theorem}{Theorem}[section]
\newtheorem{problem}{Problem}
\newtheorem{proposition}{Proposition}[section]
\newtheorem{lemma}{Lemma}[section]
\newtheorem{corollary}[theorem]{Corollary}
\newtheorem{example}{Example}[section]
\newtheorem{definition}[problem]{Definition}
\newcommand{\BEQA}{\begin{eqnarray}}
\newcommand{\EEQA}{\end{eqnarray}}
\newcommand{\define}{\stackrel{\triangle}{=}}
\theoremstyle{remark}
\newtheorem{rem}{Remark}

% Marks the beginning of the document
\begin{document}
\bibliographystyle{IEEEtran}
\vspace{3cm}

\title{Assignment 2}
\author{EE24BTECH11005 - Arjun Pavanje}
\maketitle
\newpage
\bigskip

\renewcommand{\thefigure}{\theenumi}
\renewcommand{\thetable}{\theenumi}
\section*{A. Fill in the Blanks}
\begin{enumerate}
\item Let $a,b,c$ be positive real numbers. Let
$\theta = \tan ^{-1}\brak{\sqrt{\frac{a\brak{a+b+c}}{bc}}} + \tan ^{-1}\brak{\sqrt{\frac{b\brak{a+b+c}}{ca}}} + \tan ^{-1}\brak{\sqrt{\frac{c\brak{a+b+c}}{ab}}} $
Then $\tan \brak{\theta}= \rule{2cm}{0.1pt}$ 
\hfill \brak{1981-2Marks}\\\\
\item The numerical value of $\tan \cbrak{ 2\tan ^{-1}\brak{\frac{1}{5}}-\frac{\pi}{4}}$ is equal to \rule{2cm}{0.1pt}
\hfill \brak{1984-2Marks}\\\\
\item The greater of the two angles $A = 2 \tan ^{-1}\brak{2\sqrt{2}-1}$ and $B = 3\sin ^{-1}\brak{\frac{1}{3}} + \sin ^{-1}\brak{\frac{3}{5}}$ is \rule{2cm}{0.1pt}
\hfill \brak{1989-2Marks}\\\\
\end{enumerate}
\section*{C. MCQs with One Correct Answer}
\begin{enumerate}
	\item The value of $\tan \sbrak{ \cos ^{-1}\brak{\frac{4}{5}}+\tan^{-1}\brak{\frac{2}{3}} }$ is
\hfill \brak{1983-1Mark}
\begin{multicols}{2}
\begin{enumerate}
\item $\frac{6}{17}$
\item $\frac{7}{16}$
\columnbreak
\item $\frac{16}{7}$
\item None
\end{enumerate}
\end{multicols}
\item If we consider only the principle values of the inverse trigonometric functions then the value of 
$$\tan \brak{\cos ^{-1}\brak{\frac{1}{5\sqrt{2}}}-\sin ^{-1}\brak{\frac{4}{\sqrt{17}}}}$$ is
\hfill \brak{1994}
\begin{multicols}{2}
\begin{enumerate}
\item $\frac{\sqrt{29}}{3}$ 
\item $\frac{29}{3}$
\columnbreak
\item $\frac{\sqrt{3}}{29}$ 
\item $\frac{3}{29}$ 
\end{enumerate}
\end{multicols}
\item The number of real solutions of 
$$\tan ^{-1}\brak{\sqrt{x\brak{x-1}}}+\sin ^{-1}\brak{\sqrt{x^2+x+1}}=\frac{\pi}{2}$$ is 
\hfill \brak{1999-2Marks}
\begin{multicols}{2}
\begin{enumerate}
\item zero 
\item one 
\columnbreak
\item two 
\item infinite\\\\
\end{enumerate}
\end{multicols}
\item If $\sin ^{-1}\brak{x-\frac{x^2}{2}+\frac{x^3}{4}-\ldots}+ \cos ^{-1}\brak{x^2-\frac{x^4}{2}+\frac{x^6}{4}-\ldots}=\frac{\pi}{2}$ for $0<|x|<\sqrt{2}$, then x equals 
\hfill \brak{2001S}
\begin{multicols}{2}
\begin{enumerate}
\item $\frac{1}{2}$
\item $1$ 
\columnbreak
\item $-\frac{1}{2}$ 
\item $-1$\\\\
\end{enumerate}
\end{multicols}
\item The value of $x$ for which $\sin\brak{\cot^{-1}\brak{1+x}}=\cos \brak{\tan ^{-1}\brak{x}}$ is 
\hfill \brak{2004S}
\begin{multicols}{2}
\begin{enumerate}
\item $\frac{1}{2}$ 
\item $1$
\columnbreak
\item $0$ 
\item $-\frac{1}{2}$
\end{enumerate}
\end{multicols}
\item  If $0<x<1$, then \\
	$\sqrt{1+x^2}\sbrak{\cbrak{ x\cos \brak{\cot ^{-1}\brak{x}}+ \sin \brak{\cot ^{-1}\brak{x}}}^2 -1}^{\frac{1}{2}}=$is
\hfill \brak{2008}
\begin{multicols}{2}
\begin{enumerate}
\item $\frac{x}{\sqrt{1+x^2}}$ 
\item $x$
\columnbreak
\item $x\sqrt{1+x^2}$ 
\item $\sqrt{1+x^2}$
\end{enumerate}
\end{multicols}
\item The value of $\cot \brak{\sum_{n=1}^{23} \cot ^{-1}\brak{1+\sum_{k=1}^{n} 2k}}$ is
\hfill \brak{JEE Adv.2013}
\begin{multicols}{2}
\begin{enumerate}
\item $\frac{23}{25}$ 
\item $\frac{25}{23}$ 
\columnbreak
\item $\frac{23}{24}$ 
\item $\frac{24}{23}$
\end{enumerate}
\end{multicols}
\end{enumerate}
\section*{D. MCQs with One or More than One Correct}
\begin{enumerate}
\item The principal value of $\sin ^{-1}\brak{\sin \brak{\frac{2\pi}{3}}}$ is
\hfill \brak{1986-2Marks}
\begin{multicols}{2}
\begin{enumerate}
\item $-\frac{2\pi}{3}$ 
\item $\frac{2\pi}{3}$ 
\columnbreak
\item $\frac{4\pi}{3}$ 
\item none
\end{enumerate}
\end{multicols}
\item If $\alpha=3\sin ^{-1}\brak{\frac{6}{11}}$ and $\beta=3\cos ^{-1}\brak{\frac{4}{9}}$, where the inverse trigonometric functions take only the principal values, then the correct option(s) is(are)
\hfill \brak{JEE Adv.2015}
\begin{multicols}{2}
\begin{enumerate} 
\item $\cos \brak{\beta}>0$ 
\item $\sin \brak{\beta}<0$
\columnbreak
\item $\cos \brak{\alpha + \beta} > 0$ 
\item $\cos \brak{\alpha}<0$
\end{enumerate}
\end{multicols}
\item For non-negative integers $n$, let \\\\
$f(n)= \frac{\sum_{k=0}^{n} \sin \brak{\frac{k+1}{n+2}\pi}\sin \brak{\frac{k+2}{n+2}\pi}}{\sum_{k=0}^{n} \sin ^2 \brak{\frac{k+1}{n+2}\pi}}$\\\\
Assuming $\cos ^{-1}\brak{x}$ takes values in $[0,\pi]$, which of the following options is/are correct
\hfill \brak{JEE Adv.2019}
\begin{multicols}{2}
\begin{enumerate}
\item $\lim_{n \rightarrow \infty)} f\brak{n} = \frac{1}{2}$ 
\item $f\brak{4}=\frac{\sqrt{3}}{2}$
\columnbreak
\item If $\alpha = \tan \brak{\cos ^{-1}\brak{f\brak{6}}}$, then $\alpha ^2 + 2\alpha -1 =0$
\item $\sin \brak{7\cos ^{-1}\brak{f\brak{5}}}=0$
\end{enumerate}
\end{multicols}
\end{enumerate}
\section*{E. Subjective Problems}
\begin{enumerate}
\item Find the value of: $\cos \brak{2\cos ^{-1}\brak{x}+\sin ^{-1}\brak{x}}$ at $x=\frac{1}{5}$ where $0\le \cos ^{-1}\brak{x} \le \pi$ and $-\frac{\pi}{2}\le \sin ^{-1}\brak{x} \le\frac{\pi}{2}$
\hfill \brak{1981-2Marks}\\\\
\item Find all the solution of $4\cos ^2\brak{x}\sin \brak{x}-2\sin ^2 \brak{x} = 3\sin \brak{x}$
\hfill \brak{1983-2Marks}

\end{enumerate}
\end{document}
