\let\negmedspace\undefined
\let\negthickspace\undefined
\documentclass[journal]{IEEEtran}
\usepackage[a5paper, margin=10mm, onecolumn]{geometry}
\usepackage{lmodern} % Ensure lmodern is loaded for pdflatex
\usepackage{tfrupee} % Include tfrupee package

\setlength{\headheight}{1cm} % Set the height of the header box
\setlength{\headsep}{0mm}     % Set the distance between the header box and the top of the text
\usepackage{float}
\usepackage{gvv-book}
\usepackage{gvv}
\usepackage{cite}
\usepackage{amsmath,amssymb,amsfonts,amsthm}
\usepackage{algorithmic}
\usepackage{graphicx}
\usepackage{textcomp}
\usepackage{xcolor}
\usepackage{txfonts}
\usepackage{listings}
\usepackage{enumitem}
\usepackage{mathtools}
\usepackage{gensymb}
\usepackage{comment}
\usepackage[breaklinks=true]{hyperref}
\usepackage{tkz-euclide} 
\usepackage{listings}
\usepackage{gvv}                                        
\def\inputGnumericTable{}                                 
\usepackage[latin1]{inputenc}                                
\usepackage{color}                                            
\usepackage{array}                                            
\usepackage{longtable}                                       
\usepackage{calc}                                             
\usepackage{multirow}                                         
\usepackage{hhline}                                           
\usepackage{ifthen}                                           
\usepackage{lscape}
\usepackage{listings}
\begin{document}

\bibliographystyle{IEEEtran}
\vspace{3cm}
	
\title{EE 2012 14-26}
\author{EE24BTECH11005 - Arjun Pavanje}
% \maketitle
% \newpage
% \bigskip

{\let\newpage\relax\maketitle}
\begin{enumerate}
\setcounter{enumi}{13}
	\item With the initial condition $x\brak{1}=0.5$, the solution of the differential equation 
		\begin{align*}
			t\frac{dx}{dt}+x=t
		\end{align*}
		\begin{enumerate}
				\begin{multicols}{2}
				\item $x=t-\frac{1}{2}$
						\columnbreak
					\item $x=t^2-\frac{1}{2}$
				\end{multicols}
				\begin{multicols}{2}
				\item $x=\frac{t^2}{2}$
				\columnbreak
			\item $x=\frac{t}{2}$

				\end{multicols}
		\end{enumerate}
	\item The unilateral laplace transform of $f\brak{t}=\frac{1}{s^2+s+1}$is
				\begin{enumerate}
				\begin{multicols}{2}
				\item $-\frac{s}{{s^2+s+1}^2}$
				\columnbreak
			\item $-\frac{2s-1}{{s^2+s+1}^2}$

				\end{multicols}
				\begin{multicols}{2}
				\item $\frac{s}{{s^2+s+1}^2}$

					\columnbreak
				\item$\frac{2s-1}{{s^2+s+1}^2}$

				\end{multicols}
		\end{enumerate}
	\item The average power deliver to an impedence $\brak{4-j3}\omega$ by a current $5\cos{\brak{100\pi t+100}}$
\begin{enumerate}
	\begin{multicols}{2}
		\item  $44.2 W$
		 \columnbreak
		\item  $50W$
    \end{multicols}
    \begin{multicols}{2}
	\item $62.5W$ 
    \columnbreak
\item $125W$
    \end{multicols}
\end{enumerate}
\item In the following figure, $C_1$ and $C_2$ are ideal capacitors. $C_1$ has been charged to $12V$ before the ideal switch $S$ is closed at $t=0$. The current $i\brak{t}$ for all $t$ is
\begin{figure}[H]
			\centering
			\includegraphics[scale=0.75]{figs/q17.png}
			\label{stemplot}
		\end{figure}
	\begin{enumerate}
    \begin{multicols}{2}
        \item zero
    \columnbreak
        \item a step function
    \end{multicols}
    \begin{multicols}{2}
        \item an exponentially decaying function
    \columnbreak
        \item an impulse function
    \end{multicols}
\end{enumerate}
\item The $i-v$ characteristics of the diode in the circuit given below are
	\begin{align*}
	\begin{cases}
		\frac{v-0.7}{500}A, &v\ge0.7V\\
		0A, v<0.7V
	\end{cases}
	\end{align*}
\begin{figure}[H]
			\centering
			\includegraphics[scale=0.75]{figs/q18.png}
			\label{stemplot}
		\end{figure}
		The current in the circuit is,
	\begin{enumerate}
    \begin{multicols}{2}
	\item $10mA$
    \columnbreak
\item $9.3mA$
    \end{multicols}
    \begin{multicols}{2}
	\item $6.67mA$
    \columnbreak
\item $6.2mA$
    \end{multicols}
\end{enumerate}
\item The output $Y$ of a $2-bit$ comparator is logic $1$ whenever the $2-bit$ input $A$ is greater than the $2-bit$ input $B$. The number of combinations for which the output is logic $1$.is
\begin{enumerate}
    \begin{multicols}{2}
	\item $4$
    \columnbreak
\item $6$
    \end{multicols}
    \begin{multicols}{2}
	\item $8$
    \columnbreak
\item $10$
    \end{multicols}
\end{enumerate}
\item Consider the given circuit,\\
\begin{figure}[H]
			\centering
			\includegraphics[scale=0.75]{figs/q20.png}
			\label{stemplot}
		\end{figure}
	In this circuit, the race around 
	\begin{enumerate}
    \begin{multicols}{2}
        \item Does not occur 
    \columnbreak
        \item Occurs when $CLK=0$
    \end{multicols}
    \begin{multicols}{2}
        \item Occurs when $CLK=0$ and $A=B=1$
    \columnbreak
        \item Occurs when $CLK=1$ and $A=B=0$
    \end{multicols}
\end{enumerate}
\item The figure shoes a two-generator system supplying a load of $P_D=40MW$, connected at bus $2$\\
\begin{figure}[H]
			\centering
			\includegraphics[scale=0.75]{figs/q21.png}
			\label{stemplot}
		\end{figure}
	The fuel costs of generators $G_1$, $G_2$ are:\\
$C_1\brak{P_{G1}}=10,000Rs/MWh$ and $C_2\brak{P_{G2}}=12,500Rs/MWh$\\
and the loss in the line $P_{loss\brak{pu}}=0.5P_{G1\brak{pu}}^2$, where the loss coeffecient is specified in pu on a $100MVA$ base.  The most economic power generation schedule in $MW$ is,
	\begin{enumerate}
    \begin{multicols}{2}
	\item $P_{G1}=20,P_{G2}=22$
    \columnbreak
        \item $P_{G1}=22,P_{G2}=20$
    \end{multicols}
    \begin{multicols}{2}
        \item $P_{G1}=20,P_{G2}=20$
    \columnbreak
        \item$P_{G1}=0,P_{G2}=40$
	\end{multicols}
\end{enumerate}
\item The sequence of components in a fault current are as follows: $I_{\text{positive}}=j1.5pu,I_{\text{negative}}=-j0.5pu, I_{\text{zero}}=-j1pu$ . The type of fault in the system is
	\begin{enumerate}
    \begin{multicols}{2}
	\item $LG$
    \columnbreak
\item $LL$
    \end{multicols}
    \begin{multicols}{2}
	\item $LLG$
    \columnbreak
\item $LLLG$
    \end{multicols}
\end{enumerate}
\item A half-controlled single-phase bridge rectifier is supplying an $R-L$ load. It is operated at a firing angle $\alpha$ and the load current is continuous. The fraction of cycle that the freewheeling diode conducts is
	\begin{enumerate}
    \begin{multicols}{2}
	\item $\frac{1}{2}$
    \columnbreak
\item $1-\frac{\alpha}{\pi}$
    \end{multicols}
    \begin{multicols}{2}
	\item $\frac{\alpha}{2\pi}$
    \columnbreak
\item $\frac{\alpha}{\pi}$
    \end{multicols}
\end{enumerate}

\item The typical ratio of latching current to holding current in a $20 A$ thyristor is
	\begin{enumerate}
    \begin{multicols}{2}
        \item $5.0$
    \columnbreak
        \item $2.0$
    \end{multicols}
    \begin{multicols}{2}
        \item $1.0$
    \columnbreak
        \item $0.5$
    \end{multicols}
\end{enumerate}
\item For the circuit shown in the figure, the voltage and current expressions are
	\begin{align*}
		&v\brak{t}=E_1\sin{\omega t}+E_3\sin{3\omega t}\\
		&i\brak{t}=I_1\sin{\omega t-\phi_1}+I_3\sin{3\omega t-\phi_3}+I_5\sin{5\omega t}
	\end{align*}
\begin{figure}[H]
			\centering
			\includegraphics[scale=0.75]{figs/q25.png}
			\label{stemplot}
		\end{figure}
	The average power measured by the wattmeter is
\begin{enumerate}
    \begin{multicols}{2}
	\item $\frac{1}{2}E_1I_1\cos{\phi_1}$
        \columnbreak
	\item $\frac{1}{2}\sbrak{E_1I_1\cos{\phi_1}+E_1I_3\cos{\phi_3}+E_1I_5}$
    \end{multicols}
    \begin{multicols}{2}
        \item $\frac{1}{2}\sbrak{E_1I_1\cos{\phi_1}+E_3I_3\cos{\phi_3}}$ 
        \columnbreak
        \item $\frac{1}{2}\sbrak{E_1I_1\cos{\phi_1}+E_3I_1\cos{\phi_1}}$
    \end{multicols}
\end{enumerate}
\item Given that
	\begin{align*}
		\vec{A}=\myvec{-5&-3\\2&0},\vec{I}=\myvec{1&0\\0&1}
	\end{align*}
	the value of $\vec{A}^3$ is
\begin{enumerate}
    \begin{multicols}{2}
	\item $15\vec{A}+12\vec{I}$
        \columnbreak
        \item $19\vec{A}+30\vec{I}$
    \end{multicols}
    \begin{multicols}{2}
        \item $17\vec{A}+15\vec{I}$
        \columnbreak
        \item $17\vec{A}+21\vec{I}$
    \end{multicols}
\end{enumerate}
\end{enumerate}
\end{document}
