\let\negmedspace\undefined
\let\negthickspace\undefined
\documentclass[journal]{IEEEtran}
\usepackage[a5paper, margin=10mm, onecolumn]{geometry}
\usepackage{lmodern} % Ensure lmodern is loaded for pdflatex
\usepackage{tfrupee} % Include tfrupee package

\setlength{\headheight}{1cm} % Set the height of the header box
\setlength{\headsep}{0mm}     % Set the distance between the header box and the top of the text
\usepackage{float}
\usepackage{gvv-book}
\usepackage{gvv}
\usepackage{cite}
\usepackage{amsmath,amssymb,amsfonts,amsthm}
\usepackage{algorithmic}
\usepackage{graphicx}
\usepackage{textcomp}
\usepackage{xcolor}
\usepackage{txfonts}
\usepackage{listings}
\usepackage{enumitem}
\usepackage{mathtools}
\usepackage{gensymb}
\usepackage{comment}
\usepackage[breaklinks=true]{hyperref}
\usepackage{tkz-euclide} 
\usepackage{listings}
\usepackage{gvv}                                        
\def\inputGnumericTable{}                                 
\usepackage[latin1]{inputenc}                                
\usepackage{color}                                            
\usepackage{array}                                            
\usepackage{longtable}                                       
\usepackage{calc}                                             
\usepackage{multirow}                                         
\usepackage{hhline}                                           
\usepackage{ifthen}                                           
\usepackage{lscape}
\usepackage{listings}
\begin{document}

\bibliographystyle{IEEEtran}
\vspace{3cm}
	
\title{XE 2008 18-34}
\author{EE24BTECH11005 - Arjun Pavanje}
% \maketitle
% \newpage
% \bigskip

{\let\newpage\relax\maketitle}
\begin{enumerate}
\setcounter{enumi}{17}
	\item For the initial value problem,  $\frac{dy}{dx}+y=0, y\brak{0}=1, y_1$ is the computed value of $y$ at $x=0.2$ obtained by using Euler's method with step size $h=0.1$. Then
		\begin{enumerate}
				\begin{multicols}{2}
				\item $y_1 < e^{-0.2}$
						\columnbreak
					\item $e^{-0.2}<y_1<1$
				\end{multicols}
				\begin{multicols}{2}
					\item $1<y_1$
				\columnbreak
			\item $y_1=e^{-0.2}$

				\end{multicols}
		\end{enumerate}
	\item Consider the initial value problem $\frac{dy}{dx}=y+x$ with $y\brak{0}=2$. The value of $y\brak{0.1}$ obtained using the fourth Runge-Kutta method with step size $h=0.1$ is
				\begin{enumerate}
				\begin{multicols}{2}
			\item $2.20000$
				\columnbreak
			\item $2.21500$
				\end{multicols}
				\begin{multicols}{2}
				\item $2.21551$
					\columnbreak
				\item $2.21576$ 
				\end{multicols}
		\end{enumerate}
	\item The following table gives a function $f\brak{x}$ vs $x$
\begin{table}[h!]
\centering
\begin{tabular}{|c|c|c|c|c|c|}
\hline
	$x$ & $0$ & $1$ & $2$ & $3$ & $4$ \\ \hline
	$f\brak{x}$ & $1.0$ & $3.7$ & $6.5$ & $9.3$ & $12.1$ \\ \hline
\end{tabular}
\end{table}
		The best fit of a straight line for the above data point, using a least square method error is
\begin{enumerate}
	\begin{multicols}{2}
		\item  $2.75x+0.55$
		 \columnbreak
		\item  $2.80x+0.80$
    \end{multicols}
    \begin{multicols}{2}
	\item $3.10x+0.85$ 
    \columnbreak
\item $2.78x+0.96$
    \end{multicols}
\end{enumerate}
\item Consider the following part of a Fortran $90$ function\\
\begin{lstlisting}[language=Fortran]
    INTEGER FUNCTION RESULT (X) 
        INTEGER:: X 
        VALUE =1 
        DO 
            IF (X == 0) EXIT 
            TERM = MOD (X,10) 
            VALUE = VALUE*TERM 
            X = X/10 
        END DO 
        RESULT = VALUE 
    END FUNCTION RESULT
\end{lstlisting}

	If the above function is called with an integer $X=123$, the value returned by the function will be
	\begin{enumerate}
    \begin{multicols}{2}
        \item $0$
    \columnbreak
        \item $6$
    \end{multicols}
    \begin{multicols}{2}
        \item $9$
    \columnbreak
        \item $321$
    \end{multicols}
\end{enumerate}
\item Consider the following part of a Fortran $90$ function\\
\begin{lstlisting}[language=Fortran]
PROGRAM CHECK-CYCLE
    DO I=1,10,2
        IF (MOD(I,3)==0)CYCLE
        PRINT *,I
    END DO
    TERM = MOD(X,10)
END PROGRAM CHECK-CYCLE
\end{lstlisting}

	The value returned by the program will be
	\begin{enumerate}
    \begin{multicols}{2}
	\item $\myvec{1\\5\\7}$
    \columnbreak
\item $\myvec{1\\3\\5}$
    \end{multicols}
    \begin{multicols}{2}
	\item $\myvec{1\\3\\7}$
    \columnbreak
\item $\myvec{3\\5\\7}$
    \end{multicols}
\end{enumerate}
\item $P,Q,R,S$ are segments of a Fortran $90$ code
\begin{lstlisting}[language=Fortran]
(P) IF (A > B) P=Q

(Q) SUBROUTINE SWAP(A,B)
    INTEGER, INTENT(IN)::A,B
    TEMP A
    A=B
    B=TEMP
    END SUBROUTINE SWAP
(R) IF (A/=B) X =Y-Z
    ELSE
        X=Y+Z
    ENDIF
(S) DO I=1,N,3
        C(I)=A(I)+B(I)
    END DO
\end{lstlisting}
Which segments have syntax error
	\begin{enumerate}
    \begin{multicols}{2}
        \item $P,Q$
    \columnbreak
        \item $Q,R$
    \end{multicols}
    \begin{multicols}{2}
        \item $R,S$
    \columnbreak
        \item $P,S$
    \end{multicols}
\end{enumerate}
\item A fortran-$90$ subroutine for Gauss-Siedel Method to solve a set of N simultaneous equations $ \sbrak{A} \sbrak{X}= \sbrak{C}$ is given below,
\begin{lstlisting}[language=Fortran]
SUBROUTINE SIEDEL(A,C,X,N,IMAX)
REAL :: SUM
REAL, DIMENSION(N,N) :: A
REAL, DIMENSION(N)::C,X
    DO K=1,IMAX 
        DO I=1,N
            SUM=0.0
            DO J =1, N
                IF (I/=J) THEN
                    SUM = SUM + A (I,J)*X(J)
                ENDIF
            ENDDO
            *****
        ENDDO
    ENDDO
END SUBROUTINE SIEDEL
\end{lstlisting}
The missing segment in the program indicated by ***** is,
\begin{enumerate}
    \begin{multicols}{2}
	\item $X\brak{I}=C\brak{I}+SUM$
    \columnbreak
\item $X\brak{I}=C\brak{I}-SUM$
    \end{multicols}
    \begin{multicols}{2}
	\item $X\brak{I}=\brak{C\brak{I}+SUM}/A\brak{I,I}$
    \columnbreak
\item $X\brak{I}=\brak{C\brak{I}-SUM}A\brak{I,I}$
    \end{multicols}
\end{enumerate}
\item What is the result of the following C program
\begin{lstlisting}[language=C]
int main(){
    int i, sum=0;
    for(i=0;i<25;i++){
        if(i>10) continue;
    }
    printf("%d\n",sum);
    return 1;
}
\end{lstlisting}

	\begin{enumerate}
    \begin{multicols}{2}
        \item $25$
    \columnbreak
        \item $45$
    \end{multicols}
    \begin{multicols}{2}
        \item $55$
    \columnbreak
        \item $325$
    \end{multicols}
\end{enumerate}
\item Consider the following C code
\begin{lstlisting}[language=C]
    int x=1,y=5,z;
    z=x++<<--y;
\end{lstlisting}

	\begin{enumerate}
    \begin{multicols}{2}
        \item $2,4,16$
    \columnbreak
        \item $2,4,32$
    \end{multicols}
    \begin{multicols}{2}
        \item $2,4,64$
    \columnbreak
        \item $1,5,32$
    \end{multicols}
\end{enumerate}
\item A two dimensional array is declared $int$ $num[3][3]$. Then the result of expression $*(num+1)$ is\\\\
	\begin{enumerate}
    \begin{multicols}{2}
	\item The value of $num[1][0]$
    \columnbreak
\item The value of $num[0][1]$
    \end{multicols}
    \begin{multicols}{2}
	\item The address of $num[1][0]$
    \columnbreak
\item The address of $num[0][1]$
    \end{multicols}
\end{enumerate}
\item A C function named $func$ is defined as follows is
\begin{lstlisting}[language=C]
    int func(int a, int b){
        if((a==1)||(b==0)||(a==b))return 1;
        return func(a-1,b)+func(a-1,b-1);
    }
\end{lstlisting}
	What is the result $func\brak{4,2}$
	\begin{enumerate}
    \begin{multicols}{2}
	\item $12$
    \columnbreak
\item $6$
    \end{multicols}
    \begin{multicols}{2}
	\item $3$
    \columnbreak
\item $1$
    \end{multicols}
\end{enumerate}
\textbf{Common Data for Questions 29 and 30:}\\
The following table gives the values of a function $f$ at three distant points 
\begin{table}[h!]
\centering
\begin{tabular}{|c|c|c|c|}
\hline
$x$ & $0.5$ & $0.6$ & $0.7$ \\ \hline
	$f\brak{x}$ & $0.4794$ & $0.5646$ & $0.6442$ \\ \hline
\end{tabular}
\end{table}

\item The value of $f^{\prime}\brak{x}$ at $x=0.5$ accurate upto two decimanl points, is
	\begin{enumerate}
    \begin{multicols}{2}
        \item $0.82$
    \columnbreak
        \item $0.85$
    \end{multicols}
    \begin{multicols}{2}
        \item $0.88$
    \columnbreak
        \item $30.91$
    \end{multicols}
\end{enumerate}
\item The value of $f\brak{x}$ at $x=0.55$ obtained using Newton's interpolation formula, is
	\begin{enumerate}
    \begin{multicols}{2}
        \item $0.5626$
    \columnbreak
        \item $0.5227$
    \end{multicols}
    \begin{multicols}{2}
        \item $0.4847$
    \columnbreak
        \item $0.4749$
    \end{multicols}
\end{enumerate}
\textbf{Statement for Linked Answer Questions 31 and 32:}\\
A modified Newton-Raphson method is used to find the roots of an equation $f\brak{x}=0$ which has multiple zeros at some point $x=p$ in the interval $\sbrak{a,b}$. If the multiplicity $M$ of the root is known in advance, an interative procedure for determining $p$ is given by, 
\begin{align*}
	p_{k+1}=p_k-M\frac{f\brak{p_k}}{f^{\prime}\brak{p_k}} \text{for} k=0,1,2,\dots
\end{align*}
\item The equation $f\brak{x}=x^3-1.8x^2-1.35x+2.7=0$ is known to have a multiple root in the interval $\sbrak{1,2}$. Starting with an initial guess $x_0=1.0$ in modified Newton-Ralphson method, the root, correct upto $3$ decimal places is,
	\begin{enumerate}
    \begin{multicols}{2}
        \item $1.500$
        \columnbreak
        \item $1.200$
    \end{multicols}
    \begin{multicols}{2}
        \item $1.578$
        \columnbreak
        \item $1.495$
    \end{multicols}
\end{enumerate}
\item The root of derivative of $f\brak{x}$ in the same interval is,
\begin{enumerate}
    \begin{multicols}{2}
        \item $1.500$
        \columnbreak
        \item $1.200$
    \end{multicols}
    \begin{multicols}{2}
        \item $1.578$
        \columnbreak
        \item $1.495$
    \end{multicols}
\end{enumerate}
\textbf{Statement for Linked Answer Questions 33 and 34:}\\
The values of a function $f\brak{x}$ at four discrete points are as follows,
\begin{table}[h!]
\centering
\begin{tabular}{|c|c|c|c|c|}
\hline
	$x$ & $0$ & $1$ & $3$ & $4$ \\ \hline
	$f\brak{x}$ & $-12$ & $0$ & $6$ & $12$\\ \hline
\end{tabular}
\end{table}
\item The function may be represented by a polynomial $P\brak{x}=\brak{x-a}R\brak{x}$, where $R\brak{x}$ is a polynomial of degree $2$, obtained by Lagrange's interpolation and $a$ is a real constant. The polynomial $R\brak{x}$ is,
\begin{enumerate}
    \begin{multicols}{2}
        \item $x^2+6x+12$
        \columnbreak
        \item $x^2+6x-12$
    \end{multicols}
    \begin{multicols}{2}
        \item $x^2-6x-12$
        \columnbreak
        \item $x^2-6x+12$
    \end{multicols}
\end{enumerate}
\item The value of the derivative of the interpolated polynomial $P\brak{x}$ at the position of its real is
\begin{enumerate}
    \begin{multicols}{2}
        \item $-6$
        \columnbreak
        \item $-4$
    \end{multicols}
    \begin{multicols}{2}
        \item $6$
        \columnbreak
        \item $7$
    \end{multicols}
\end{enumerate}
\end{enumerate}
\end{document}
